% !TEX encoding = IsoLatin2
\documentclass[Master, BBE, english]{twbook}
\usepackage[T1]{fontenc}
% Hier kann je nach Betriebssystem eine der folgenden Optionen notwendig sein, um die Umlaute korrekt wiederzugeben:
% utf8, latin, applemac
\usepackage[ansinew]{inputenc}
% Die nachfolgenden 2 Pakete stellen sonst nicht ben�tigte Features zur Verf�gung
\usepackage{blindtext,dtklogos}

\title{The thesis title}
\author{My name, BSc}
\studentnumber{0000000000}
\supervisor{Dr. Ing. My supervisor}
\secondsupervisor{Prof. Dr. Noch mehr}
\place{Vienna}
\kurzfassung{\blindtext}
\schlagworte{Schlagwort1, Schlagwort2, Schlagwort3, Schlagwort4}
\outline{\blindtext}
\keywords{Keyword1, Keyword2, Keyword3, Keyword4}
\acknowledgements{\blindtext}

\begin{document}
\maketitle

\Blinddocument

\chapter{Erste �berschrift der Ebene 1(chapter)}
\blinddocument

\blindmathpaper

\section{Erste �berschrift Tiefe 2}(section)
\blindtext 

\subsection{Erste �berschrift Tiefe 3 (subsection)}
\blindtext 

\subsubsection{Erste �berschrift Tiefe 4 (subsubsection)}
\blindtext

\chapter{Zweite �berschrift der Tiefe 1 (chapter)}
\blindtext  

\section{Zweite �berschrift Tiefe 2 (section)}
\blindtext  

\section{Zweite �berschrift Tiefe 2 (section)}
\blindtext 

\subsection{Zweite �berschrift Tiefe 3 (subsection)}
\blindtext 

\subsection{Dritte �berschrift Tiefe 3 (subsection)}
\blindtext 
 
\chapter{Zweite �berschrift Tiefe 0 (chapter)}
\blindtext 

\noindent Querverweise werden in \LaTeX{} automatisch erzeugt und verwaltet, damit sie leicht aktualisiert werden k�nnen. Hier wird zum Beispiel auf Abbildung \ref{Abb1} verwiesen.

\begin{figure}[!htbp]
\centering
\includegraphics[width=0.5\linewidth]{PICs/Buchruecken}
\caption{Beispiel f�r die Beschriftung eines Buchr�ckens.}\label{Abb1}
\end{figure}


\begin{figure}[!htbp]
\centering
\includegraphics[width=0.5\linewidth]{PICs/Buchruecken}
\caption{Beispiel f�r die Beschriftung eines Buchr�ckens.}\label{Abb3}
\end{figure}


Und hier ist ein Verweis auf Tabelle \ref{tab1}. Das gezeigte Tabellenformat ist nur ein Beispiel. Tabellen k�nnen individuell gestaltet werden.

\begin{table}[!htbp]
\centering
\begin{tabular}{| p{0.3\linewidth} | p{0.3\linewidth} | p{0.3\linewidth} |}\hline
Datum & Thema & Raum\\\hline
20.08.2008 & Graphentheorie	& HS 3.13\\
01.10.2008 & Biomathematik & HS 1.05\\\hline
\end{tabular}
\caption{Semesterplan der Lehrveranstaltung \glqq Angewandte Mathematik\grqq.}\label{tab1}
\end{table}

Hier wird auf die Formel \ref{Gl1} verwiesen.

\begin{align}
x = -\frac{p}{2}\pm\sqrt{\frac{p^2}{4}-q}\label{Gl1}
\end{align}

Literaturverweise sollten automatisch verwaltet werden, vor allem dann, wenn es viele Quellenverweise gibt. Hier wird auf \cite{Balzert:2005} und \cite{Wagner:2007,Aloyetal:1995} verwiesen. Das verwendete Zitierformat (bzw. das Format des Literaturverzeichnisses) wird entspechend den Vorgaben der Studieng�nge automatisch ausgew�hlt. Es wird dringend empfohlen, \BibTeX zu verwenden (also nicht die Literaturquellen wie in diesem Beispiel manuell im Dokument einzugeben. 

\chapter{Zweite �berschrift Tiefe 0 (chapter)}
\blindtext 

\noindent Querverweise werden in \LaTeX{} automatisch erzeugt und verwaltet, damit sie leicht aktualisiert werden k�nnen. Hier wird zum Beispiel auf Abbildung \ref{Abb1} verwiesen.

\begin{figure}[!htbp]
\centering
\includegraphics[width=0.5\linewidth]{PICs/Buchruecken}
\caption{Beispiel f�r die Beschriftung eines Buchr�ckens.}\label{Abb2}
\end{figure}

Und hier ist ein Verweis auf Tabelle \ref{tab1}. Das gezeigte Tabellenformat ist nur ein Beispiel. Tabellen k�nnen individuell gestaltet werden.

\begin{table}[!htbp]
\centering
\begin{tabular}{| p{0.3\linewidth} | p{0.3\linewidth} | p{0.3\linewidth} |}\hline
Datum & Thema & Raum\\\hline
20.08.2008 & Graphentheorie	& HS 3.13\\
01.10.2008 & Biomathematik & HS 1.05\\\hline
\end{tabular}
\caption{Semesterplan der Lehrveranstaltung \glqq Angewandte Mathematik\grqq.}\label{tab2}
\end{table}

Hier wird auf die Formel \ref{Gl1} verwiesen.

\begin{align}
x = -\frac{p}{2}\pm\sqrt{\frac{p^2}{4}-q}\label{Gl2}
\end{align}

Literaturverweise sollten automatisch verwaltet werden, vor allem dann, wenn es viele Quellenverweise gibt. Hier wird auf \cite{Balzert:2005} und \cite{Wagner:2007,Aloyetal:1995} verwiesen. Das verwendete Zitierformat (bzw. das Format des Literaturverzeichnisses) wird entspechend den Vorgaben der Studieng�nge automatisch ausgew�hlt. Es wird dringend empfohlen, \BibTeX zu verwenden (also nicht die Literaturquellen wie in diesem Beispiel manuell im Dokument einzugeben. 

\clearpage
\bibliographystyle{plain}
\begin{thebibliography}{99}
\bibitem{Balzert:2005}
H.~Balzert \newblock{\em{Lehrbuch der Objektmodellierung - Analyse und Entwurf mit der UML 2}},  2. Ausg., Elsevier GmbH, M�nchen 2005.

\bibitem{Wagner:2007}
K.W.~Wagner \newblock{\em{Performance Excellence. Der Praxisleitfaden zum effektiven Prozessmanagement}}, Hanser Fachbuch, M�nchen 2007.

\bibitem{Aloyetal:1995}
A.~Aloy, E.~Schragl, H.~Neth, A.~Donner, und A.~Kluwick \newblock{\em{Str�mungsverhalten des Atemgases bei SHFJ Jet-Laryngoskop}}
\newblock {\textsc {Der An�sthesist}}, 44:558--565, 1995.
\end{thebibliography}
\clearpage
Hallo
\clearpage

% Das Abbildungsverzeichnis
\listoffigures
\clearpage

% Das Tabellenverzeichnis
\listoftables
\clearpage

\phantomsection
\addcontentsline{toc}{chapter}{Abk�rzungsverzeichnis}
\chapter*{Abk�rzungsverzeichnis}
\begin{acronym}[XXXXX]
	\acro{ABC}[ABC]{Alphabet}
	\acro{WWW}[WWW]{world wide web}
	\acro{ROFL}[ROFL]{Rolling on floor laughing}
\end{acronym}
\end{document}